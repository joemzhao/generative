%%%%%%%%%%%%%%%%%%%%%%%%%%%%%%%%%%%%%%%%%
% Beamer Presentation
% LaTeX Template
% Version 1.0 (10/11/12)
%
% This template has been downloaded from:
% http://www.LaTeXTemplates.com
%
% License:
% CC BY-NC-SA 3.0 (http://creativecommons.org/licenses/by-nc-sa/3.0/)
%
%%%%%%%%%%%%%%%%%%%%%%%%%%%%%%%%%%%%%%%%%

%----------------------------------------------------------------------------------------
%	PACKAGES AND THEMES
%----------------------------------------------------------------------------------------

\documentclass{beamer}

\mode<presentation> {

% The Beamer class comes with a number of default slide themes
% which change the colors and layouts of slides. Below this is a list
% of all the themes, uncomment each in turn to see what they look like.

% \usetheme{default}
% \usetheme{AnnArbor}
% \usetheme{Antibes}
% \usetheme{Bergen}
% \usetheme{Berkeley}
% \usetheme{Berlin}
\usetheme{Boadilla}
% \usetheme{CambridgeUS}
% \usetheme{Copenhagen}
% \usetheme{Darmstadt}
% \usetheme{Dresden}
% \usetheme{Frankfurt}
% \usetheme{Goettingen}
% \usetheme{Hannover}
% \usetheme{Ilmenau}
% \usetheme{JuanLesPins}
% \usetheme{Luebeck}
% \usetheme{Madrid}
% \usetheme{Malmoe}
% \usetheme{Marburg}
% \usetheme{Montpellier}
% \usetheme{PaloAlto}
% \usetheme{Pittsburgh}
% \usetheme{Rochester}
% \usetheme{Singapore}
% \usetheme{Szeged}
% \usetheme{Warsaw}

% As well as themes, the Beamer class has a number of color themes
% for any slide theme. Uncomment each of these in turn to see how it
% changes the colors of your current slide theme.

%\usecolortheme{albatross}
%\usecolortheme{beaver}
%\usecolortheme{beetle}
%\usecolortheme{crane}
%\usecolortheme{dolphin}
%\usecolortheme{dove}
%\usecolortheme{fly}
%\usecolortheme{lily}
%\usecolortheme{orchid}
%\usecolortheme{rose}
%\usecolortheme{seagull}
%\usecolortheme{seahorse}
%\usecolortheme{whale}
%\usecolortheme{wolverine}

%\setbeamertemplate{footline} % To remove the footer line in all slides uncomment this line
%\setbeamertemplate{footline}[page number] % To replace the footer line in all slides with a simple slide count uncomment this line

%\setbeamertemplate{navigation symbols}{} % To remove the navigation symbols from the bottom of all slides uncomment this line
}

\usepackage{booktabs} % Allows the use of \toprule, \midrule and \bottomrule in tables
\usepackage{graphicx}
\usepackage{subcaption}
%----------------------------------------------------------------------------------------
%	TITLE PAGE
%----------------------------------------------------------------------------------------

\title[\tiny{Response in Discrete Space GANs}]{Response Generation in Discrete Space Using GANs} % The short title appears at the bottom of every slide, the full title is only on the title page

\author[M. Zhao et F. Mi]{\underline{Mengjie Zhao}  \and Fei Mi } % Your name
\institute[] % Your institution as it will appear on the bottom of every slide, may be shorthand to save space
{
\footnotesize{Artificial Intelligence Laboratory (LIA)}
\date{\today}
\\\medskip
\medskip
\medskip
\medskip
\medskip
\medskip
% \footnotesize{\'Ecole polytechnique f\'ed\'erale de Lausanne} \\ % Your institution for the title page
\medskip
% \textit{john@smith.com} % Your email address
\vspace{-1cm}
\begin{figure}
\includegraphics[width=0.2\linewidth]{imgs/epfl1.eps}
\end{figure}
}

 % Date, can be changed to a custom date

\begin{document}
\begin{frame}
\titlepage % Print the title page as the first slide
\end{frame}

\begin{frame}
\frametitle{Overview} % Table of contents slide, comment this block out to remove it
\tableofcontents % Throughout your presentation, if you choose to use \section{} and \subsection{} commands, these will automatically be printed on this slide as an overview of your presentation
\end{frame}

%----------------------------------------------------------------------------------------
%	PRESENTATION SLIDES
%----------------------------------------------------------------------------------------

%------------------------------------------------
\section{A quick reminder of GANs} % Sections can be created in order to organize your presentation into discrete blocks, all sections and subsections are automatically printed in the table of contents as an overview of the talk
%------------------------------------------------

\begin{frame}
\frametitle{Application of GAN}
\begin{figure}
\includegraphics[width=0.8\linewidth]{imgs/w1_g_model.png}
\end{figure}
\hspace{7cm}\tiny{src: http://torch.ch/blog/2015/11/13/gan.html}
\end{frame}


%------------------------------------------------

%------------------------------------------------

\begin{frame}
\frametitle{Performance on images (3-conv-layer CNN)}
\begin{figure}[!htb]
\minipage{0.3\textwidth}
  \includegraphics[width=\linewidth]{imgs/1_face.png}
  \caption*{Noise}\label{fig:awesome_image1}
\endminipage\hfill
\minipage{0.3\textwidth}
  \includegraphics[width=\linewidth]{imgs/2_face.png}
  \caption*{Epoch 1}\label{fig:awesome_image2}
\endminipage\hfill
\minipage{0.3\textwidth}%
  \includegraphics[width=\linewidth]{imgs/3_face.png}
  \caption*{Epoch 3}\label{fig:awesome_image3}
\endminipage
\end{figure}
\hspace{7cm}\tiny\emph{Dataset: Labeled Faces in the Wild Home, UMass}
\end{frame}

%------------------------------------------------

% %------------------------------------------------

% \begin{frame}
% \frametitle{To be more formal}
% \begin{itemize}
% \item Define $p_{data}$ as the distribution where data generated from
% \item Input noise $\textbf{z}$ from noise distribution $p_z(z)$
% \item Generator: $G(z;\theta_{g}) \Rightarrow \textbf{x}$
% \item Discriminator: $D(\textbf{x}; \theta_{d})$ the probability $\textbf{x}$ is from $p_{data}$
% \item G and D are trained simultaneously.
% \item Two-player minimax game with value function $V(G, D)$:
% \end{itemize}
% \begin{figure}
% \includegraphics[width=0.8\linewidth]{imgs/game}
% \end{figure}
% \end{frame}

% %------------------------------------------------

% %------------------------------------------------

% \begin{frame}
% \frametitle{To be more cartoon...}
% \begin{figure}
% \includegraphics[width=\linewidth]{imgs/gan_numeric}
% \end{figure}
% \vspace{1cm}
% \hspace{9cm}\tiny{(Goodfellow et al. 2014)}
% \end{frame}
% %------------------------------------------------

%------------------------------------------------
% \begin{frame}
% \frametitle{Performance of a toy GAN (full connected MLPs)}
% \begin{figure}
% \centering
% \begin{minipage}{.5\textwidth}
%   \centering
%   \includegraphics[width=.62\linewidth]{imgs/gan_generated_img_e_50.png}
%   \captionof{figure}*{Generated letters from G (50 epochs)}
%   \label{fig:test1}
% \end{minipage}%
% \begin{minipage}{.5\textwidth}
%   \centering
%   \includegraphics[width=.83\linewidth]{imgs/loss_d.eps}
%   \captionof{figure}*{Loss of D (50 epochs)}
%   \label{fig:test2}
% \end{minipage}
% \end{figure}
% \end{frame}
%------------------------------------------------



% \begin{frame}
% \frametitle{Bullet Points}
% \begin{itemize}
% \item Lorem ipsum dolor sit amet, consectetur adipiscing elit
% \item Aliquam blandit faucibus nisi, sit amet dapibus enim tempus eu
% \item Nulla commodo, erat quis gravida posuere, elit lacus lobortis est, quis porttitor odio mauris at libero
% \item Nam cursus est eget velit posuere pellentesque
% \item Vestibulum faucibus velit a augue condimentum quis convallis nulla gravida
% \end{itemize}
% \end{frame}

%------------------------------------------------

% \begin{frame}
% \frametitle{Blocks of Highlighted Text}
% \begin{block}{Block 1}
% Lorem ipsum dolor sit amet, consectetur adipiscing elit. Integer lectus nisl, ultricies in feugiat rutrum, porttitor sit amet augue. Aliquam ut tortor mauris. Sed volutpat ante purus, quis accumsan dolor.
% \end{block}

% \begin{block}{Block 2}
% Pellentesque sed tellus purus. Class aptent taciti sociosqu ad litora torquent per conubia nostra, per inceptos himenaeos. Vestibulum quis magna at risus dictum tempor eu vitae velit.
% \end{block}

% \begin{block}{Block 3}
% Suspendisse tincidunt sagittis gravida. Curabitur condimentum, enim sed venenatis rutrum, ipsum neque consectetur orci, sed blandit justo nisi ac lacus.
% \end{block}
% \end{frame}

%------------------------------------------------

% \begin{frame}
% \frametitle{Multiple Columns}
% \begin{columns}[c] % The "c" option specifies centered vertical alignment while the "t" option is used for top vertical alignment

% \column{.45\textwidth} % Left column and width
% \textbf{Heading}
% \begin{enumerate}
% \item Statement
% \item Explanation
% \item Example
% \end{enumerate}

% \column{.5\textwidth} % Right column and width
% Lorem ipsum dolor sit amet, consectetur adipiscing elit. Integer lectus nisl, ultricies in feugiat rutrum, porttitor sit amet augue. Aliquam ut tortor mauris. Sed volutpat ante purus, quis accumsan dolor.

% % \end{columns}
% % \end{frame}

% %------------------------------------------------
% \section{Generating in discrete space?}
% %------------------------------------------------

\begin{frame}
\frametitle{GAN on text}
\begin{itemize}
\item Training the model 
  \begin{itemize}
    \item Training on continuous space
    \item Using \textbf{policy iteration}
  \end{itemize} 
\item Two major types
	\begin{itemize}
		\item Text generation
		\item Response and dialog generation
		% a cntinuous distribution that can approximate
		% categorical samples

	\end{itemize}
\end{itemize}
\end{frame}

%------------------------------------------------
\begin{frame}
\frametitle{Overview} 
\tableofcontents 
\end{frame}
%------------------------------------------------

%------------------------------------------------
\section{Our current framework} 
%------------------------------------------------

\begin{frame}
\frametitle{Our current framework}
\begin{figure}
\centering
\includegraphics[width=\linewidth]{imgs/v2.eps}
\end{figure}
\end{frame}

%------------------------------------------------
%------------------------------------------------
\begin{frame}
\frametitle{The encoder-decoder model}
\begin{itemize}
\item Taking sequences as input and generates sequence as output:
\begin{figure}
\centering
\hspace{-1cm}\includegraphics[width=0.85\linewidth]{imgs/en-de.png}
\end{figure}
\hspace{9cm}\tiny{(C. Olah 2015)}
\end{itemize}
\begin{itemize}
\item Sample results (TBBT dataset):
\begin{itemize}
\item \emph{what is your name ? -- my name is siri .}
\item \emph{do you know switzerland ? -- oh , of course .}
\item \emph{i think you are correct . -- so why are you saying that ?}
\item \emph{can you drive me home ? -- uh , sure .}
\end{itemize}
\end{itemize}
\end{frame}
%------------------------------------------------
%------------------------------------------------
\begin{frame}
\frametitle{Convolutional fusion - old method}
A': The representation of an answer consisting of time dependency words

  \begin{figure}
      \centering
      \hspace{-1cm}\includegraphics[width=0.75\linewidth]{imgs/idea}
  \end{figure}
\end{frame}
%------------------------------------------------

%------------------------------------------------
\begin{frame}
\frametitle{Fusing in another latent space}
\begin{itemize}
\item For each candidate answer from beam search, we express its embedding in another latent space:
  \begin{figure}
      \centering
      \hspace{-1cm}\includegraphics[width=0.85\linewidth, height=0.65\linewidth]{imgs/fuse}
  \end{figure}
\end{itemize}
\end{frame}
%------------------------------------------------

%------------------------------------------------
\begin{frame}
\frametitle{Fusing in another latent space}
\begin{itemize}
\item Take weigted average of the new latent expressions of all candidates from beam search:
      \begin{itemize}
      \item Weigts are set based on score of each candidates from beam search.
      \item Trainable weights.
      \end{itemize}
      \vspace{.4cm}
\item Start adversarial training:
      \begin{itemize}
      \item The new fused expression is initial state of the LSTM in SeqGAN.
      \item Use the same embedding from the encoder-decoder model.
      \end{itemize}
\end{itemize}
\end{frame}
%------------------------------------------------

\begin{frame}
\frametitle{Major drawback of the proposed framework}
\begin{figure}
\centering
\includegraphics[width=\linewidth]{imgs/v2.eps}
\end{figure}
\begin{itemize}
\item This framework is not end-to-end due to the beam search operation.
\item We need to check if the gradient could be backpropagated to update parameters in the fusing operation. 
\end{itemize}
\end{frame}

%------------------------------------------------
\begin{frame}
\frametitle{Generated sentences by SeqGAN}
\begin{itemize}
\item Applying SeqGAN over the Big Bang Theory subtitle dataset
\item Some generated sentences:
	\begin{itemize}
		\item \emph{how was that , but that doesn `t come that waitress kisses me stop}
		\item \emph{you were my chicken ? no hello} 
		\item \emph{penny you `re just never been decided you put it `s nice to talk}
		\item \emph{{{[Howard]}} sheldon i should just do that ?}
		% a cntinuous distribution that can approximate
		% categorical samples

	\end{itemize}
% \item Reinforcement training is increasingly used for text generation
\end{itemize}
\end{frame}

%------------------------------------------------

%------------------------------------------------
\begin{frame}
\frametitle{Dialog generation (Li et al. 2017)}
\begin{itemize}
\item Use \emph{seq2seq} model as a generator to generate dialog
\item Discriminator decides the source of the dialog
\item Policy iteration
\end{itemize}
\end{frame}

%------------------------------------------------
%------------------------------------------------
\begin{frame}
\frametitle{For QA problem - a tentative framework}
A': The representation of an answer consisting of time dependency words

 	\begin{figure}
			\centering
			\hspace{-1cm}\includegraphics[width=0.75\linewidth]{imgs/idea}
	\end{figure}
\end{frame}

%------------------------------------------------
%------------------------------------------------
\begin{frame}
\frametitle{Overview} % Table of contents slide, comment this block out to remove it
\tableofcontents % Throughout your presentation, if you choose to use \section{} and \subsection{} commands, these will automatically be printed on this slide as an overview of your presentation
\end{frame}
%------------------------------------------------

% \begin{frame}
% \frametitle{Theorem}
% \begin{theorem}[Mass--energy equivalence]
% $E = mc^2$
% \end{theorem}
% \end{frame}

%------------------------------------------------

% \begin{frame}[fragile] % Need to use the fragile option when verbatim is used in the slide
% \frametitle{Verbatim}
% \begin{example}[Theorem Slide Code]
% \begin{verbatim}
% \begin{frame}
% \frametitle{Theorem}
% \begin{theorem}[Mass--energy equivalence]
% $E = mc^2$
% \end{theorem}
% \end{frame}\end{verbatim}
% \end{example}
% \end{frame}

%------------------------------------------------

% \begin{frame}
% \frametitle{Figure}
% Uncomment the code on this slide to include your own image from the same directory as the template .TeX file.
% %\begin{figure}
% %\includegraphics[width=0.8\linewidth]{test}
% %\end{figure}
% \end{frame}

%------------------------------------------------
\section{Rate of progress}
%------------------------------------------------
\begin{frame}
\frametitle{Rate of progress}
\begin{figure}
\centering
\includegraphics[width=\linewidth]{imgs/gantt}
\end{figure}
\end{frame}

%------------------------------------------------
\section{Conclusion}
%------------------------------------------------
\begin{frame}
\frametitle{Conclusion}
\begin{itemize}
% \item Adversarial training for generative models
\item In GANs, differentiable G and D are required
\item Policy iteration in text generation
\item Applying GANs in the QA problem is to be considered
\end{itemize}
\end{frame}

%------------------------------------------------

% \begin{frame}[fragile] % Need to use the fragile option when verbatim is used in the slide
% \frametitle{Citation}
% An example of the \verb|\cite| command to cite within the presentation:\\~

% This statement requires citation \cite{p1}.
% \end{frame}

%------------------------------------------------
%------------------------------------------------
\begin{frame}
\vspace{3cm}
\Huge{\centerline{That's it!}}
\begin{figure}
\hspace{-7cm}
\includegraphics[width=0.2\linewidth]{imgs/bt}
\end{figure}
\end{frame}
%---------------------------------------------------

\begin{frame}
\frametitle{References}
\footnotesize{
\begin{thebibliography}{99} % Beamer does not support BibTeX so references must be inserted manually as below
\bibitem[Goodfellow et al]{p1} Goodfellow et al. (2014)
\newblock Generative Adversarial Networks
\newblock \emph{NIPS 2014}

\bibitem[Yu et al]{p1} Yu et al. (2017)
\newblock SeqGAN: Sequence Generative Adversarial Nets with Policy Gradient 
\newblock \emph{AAAI 17}

\bibitem[Li et. al.]{p1} Li et al. (2017)
\newblock Adversarial Learning for Neural Dialogue Generation
\newblock \emph{arXiv:1701.06547}

\bibitem[Jang et. al.]{p1} Jang et al. (2016)
\newblock Categorical Reparameterization with Gumbel-Softmax
\newblock \emph{arXiv:1611.01144}


\end{thebibliography}
}
\end{frame}

%-------------------------------------

%------------------------------------------------
\begin{frame}
\frametitle{Appendix}
\begin{itemize}
\item BLEU (bilingual evaluation understudy):
	\begin{itemize}
 		\item the closer a machine translation is to a professional human translation, the better it is
 		\item evaluating the quality of text which has been machine-translated from one natural language to another. 
 	\end{itemize}
 \item GitHub: @joemzhao
 \item Likelihood:
		 \begin{figure}
			\centering
			\includegraphics[width=0.7\linewidth]{imgs/likelihood}
		\end{figure}
\end{itemize}
\end{frame}
%---------------------------------------------------

%------------------------------------------------
\begin{frame}
\frametitle{Appendix}
\begin{itemize}
\item SeqGAN 
\begin{itemize}
		\item Generator LSTM maximize the expected rewards
		\item Discriminator TextCNN minimize cross entropy
\end{itemize}
\end{itemize}
\end{frame}
%---------------------------------------------------

\end{document}